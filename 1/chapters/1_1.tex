\section{Signals and Systems} 

\paragraph{a)}

$z^t$ is an eigensignal of a LTI system. It can be shown using the input/output relation of an
LTI system (convolution).

An Eigensignal has to satisfy the following condition:

\begin{equation}
 y(t) = H\{x(t)\} = \underbrace{\lambda}_{Eigenvalue} \* \underbrace{x(t)}_{Eigensignal}
\end{equation}

Setting $x(t)$ to $z^t$ leads to the following:
\begin{equation}
 y(t) = \int_{-\infty}^{\infty} x(t-\tau ) \* h(\tau ) d \tau = 
 \int_{-\infty}^{\infty} z^{t-\tau} \* h(\tau ) d \tau = 
 \underbrace{z^t}_{Eigensignal} \* \underbrace{ \int_{-\infty}^{\infty} z^{-\tau} \* h(\tau ) d \tau }_{Eigenvalue}
\end{equation}

You can see at the equation above, that an eigensignal $z^t$ at the input of a LTI, leads
to the scaled Eigensignal at the ouput. The Eigensignal will be scaled by the Eigenvalue.


\begin{equation}
 z^t = e^{ln(z) \* t}
\end{equation}
Relation between $s$ and $z$:
\begin{equation}
 s = ln(z)
\end{equation}


\paragraph{b)}
To find the impulse response $h[n]$ of the system given by the difference equation, we perform
a z transformation of the given difference equation.

\begin{equation}
 a_1 \* Y(z) = b_2 \* X(z) \* z^{-1} + b_1 \* X(z) + b_0 + a_2 \* Y(z) z^{-1}
\end{equation}

\begin{equation}
Y(z) \* [a_1 - a_2 \* z^{-1}]= X(z) \* [b_2 \* z^{-1} + b_1] + b_0
\label{eq:sysz}
\end{equation}


As you can see, at the equation above, you cannot find an expression for the relation
$H(z) = \frac{Y(z)}{X(z)}$. I claim that the given system is time-variant and non linear, and therefore we can not
find an expression for $h[n]$ (i.e $H(Z)$).

The system is non-linear, because it does not meet the homogeneity condition ($f(a \* x) = a \* f(x)$).
Multiplying the input $X(z)$ by a factor $a$ does not lead to the output $a \* Y(z)$

To proof that the system is time-variant, we have to show that the condition
for time-invariance will not be fulfilled for any signal $x(t)$.

Time Invariance: (for any signal $x[n]$)

\begin{equation}
 y[n] = H\{x[n]\} => y[n-k] = H\{x[n-k]\}
\end{equation}

So let's proof it:

$$x[n] \laplace X(z)$$
$$x[n-k] \laplace z^{-k} X(z)$$
$$y[n] \laplace Y(z)$$
$$y[n-k] \laplace z^{-k} Y(z)$$

Replacing $x[n]$ by $x[n-k]$ in Equation \ref{eq:sysz} should lead to $y[n-k]$,
But this is a different equation:

\begin{equation}
z^{-k}Y(z) \* [a_1 - a_2 \* z^{-1}]= z^{-k}X(z) \* [b_2 \* z^{-1} + b_1] + b_0
\end{equation}

But as you can see, this equation is different than the Equation \ref{eq:sysz}.

\begin{equation}
Y(z) \* [a_1 - a_2 \* z^{-1}]= X(z) \* [b_2 \* z^{-1} + b_1] + z^{-k}b_0
\end{equation}

\paragraph{c)}

\begin{equation}
 y(t) = \int_{-\infty}^{\infty} h(t,\hat{\tau}) \* x(t-\hat{\tau}) d\hat{\tau}
\end{equation}

Let's calculate the output $\hat{h}(t,\tau)$ of the system, by inserting $x(t) = \delta(t-\tau)$
into the equation above.

\begin{equation}
 \hat{h}(t,\tau) = \int_{-\infty}^{\infty} h(t,\hat{\tau}) \* \delta(t-\tau - \hat{\tau}) d\hat{\tau}
\end{equation}

The dirac-delta $\delta(t-\tau - \hat{\tau})$ is nonzero for $\hat{\tau} = t-\tau$:

\begin{equation}
 \hat{h}(t,\tau) = h(t,t-\tau)
\end{equation}


\paragraph{d)}


The second impulse response


